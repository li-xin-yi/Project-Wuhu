\documentclass[xcolor={usenames,dvipsnames}]{beamer}
\logo{\includegraphics[height=.75cm]{logo.jpg}}
\usepackage{ctex}
\usepackage{beamerthemesplit} 
\usepackage{fontawesome}
\usetheme{Madrid}
\usepackage{graphicx}
%\newfontfamily{\FA}{FontAwesome}
\begin{document}
\setbeamertemplate{footline}{%
	\leavevmode%
	\hbox{%
		\begin{beamercolorbox}[wd=.45\paperwidth,ht=2.25ex,dp=1ex,center]{author in head/foot}%
			\usebeamerfont{section in head/foot}{自助阅读}{}{}
		\end{beamercolorbox}%
		\begin{beamercolorbox}[wd=.22\paperwidth,ht=2.25ex,dp=1ex,center]{title in head/foot}%
			\usebeamerfont{title in head/foot}\insertshorttitle
		\end{beamercolorbox}%
		\begin{beamercolorbox}[wd=.3333\paperwidth,ht=2.25ex,dp=1ex,right]{date in head/foot}%
			\usebeamerfont{date in head/foot}\insertshortdate{}\hspace*{2em}
			\insertframenumber{} / \inserttotalframenumber\hspace*{2ex}
		\end{beamercolorbox}}%
		\vskip0pt%
	}
%\newfontfamily{\FA}{FontAwesome Regular}
\title[数模组(MCM)]{\textbf{2015创新实践中心数模组招新}} % The short title appears at the bottom of every slide, the full title is only on the title page

\author[C103]{教学楼C区103} % Your name
\institute[DLUT] % Your institution as it will appear on the bottom of every slide, may be shorthand to save space
{
大连理工大学软件学院 创新实践中心 \\ % Your institution for the title page
%\medskip\textbf{}
%\textit{john@smith.com} % Your email address
SSDUT CIPPUS
}
\date{\today} % Date, can be changed to a custom date
\section{自助阅读:}
\section{请使用$\leftarrow$和$\to$键查看前后页}
\linespread{1}
\begin{frame}
\titlepage % Print the title page as the first slide
\end{frame}

\begin{frame}{\textbf{数模是什么?}}
\begin{exampleblock}{}
数学建模就是通过计算得到的结果来解释\textbf{实际问题},并接受实际的检验,来建立\textbf{数学模型}的全过程。当需要从\textbf{定量}的角度分析和研究一个实际问题时,人们就要在深入调查研究、了解对象信息、作出简化假设、分析内在规律等工作的基础上,用数学的符号和语言作表述来建立数学模型。 
\end{exampleblock}
\vspace{.5cm}
\begin{alertblock}{}
可以说,即使是完成数学试卷上一道小小的应用题,也是这样建立数学模型并求解的过程。
\end{alertblock} 
\end{frame}

\begin{frame}{\textbf{举个简单的例子}}
\begin{exampleblock}{来解决一个简单的应用题}
一个细菌培养过程在初始时刻的细菌数量为$P_0$,在$t=1h$时,测得细菌数量为$1.5P_0$。如果在$t$时刻的增长与此刻细菌数量$P(t)$成正比,求细菌数量增加至原来的3倍所需要的时间。
\end{exampleblock}

\begin{columns}
\column{.55\textwidth}
\begin{description}
 \item[设] $P'(t) = k \cdot P(t)$ 
 \item[微分方程] $\frac{{dP}}{{dt}} = kP(t)$
 \item[即]  $\frac{{dP}}{{P}} = k \cdot dt$
 \item[两边积分] $\int {\frac{{dP}}{P} = \int {kdt} } $\\
 $ \Rightarrow \ln P(t) = kt + C$
 \item[代入数值] 在$t=0$时,$P(t)=P_0$\\
 在$t=1$时,$P(t)=1.5P_0$
\end{description}
\column{.5\textwidth}
\begin{description}
	 \item[微分方程组] $
	 \left\{
	 \begin{aligned}
	 \ln P(t) &= kt + C \\
	 P(0)&=P_0 \\
	 P(1)&=1.5P_0 \\
	 \end{aligned}
	 \right.
	 $
	 \item[解得] $
	 \left\{
	 \begin{aligned}
	 k&=\ln 3-\ln 2 \\
	 C&=\ln P_0 \\
	 \end{aligned}
	 \right.
	 $
	 \item[求解方程] $\ln 3P_0 = kt + C$
	 \item[解] $ 
	 \begin{aligned}
	 t&=\ln 3/(ln3 - \ln 2) \\
	  & \approx 2.71 (h)
	 \end{aligned}$
\end{description}
\end{columns}
\end{frame}

\begin{frame}{\textbf{数模能做什么?}}
	\framesubtitle{在获得数据等信息后,通过数学建模,解决一些有趣的实际问题}
	\begin{itemize}
		\item 根据用户的浏览购买记录智能推荐商品
		\item 预测每个用户对每部电影的评分
		\item 根据抽样成分测评葡萄酒
		\item 计算人造卫星的发射速度和着陆控制策略
		\item 通过社交网络互动信息找到潜在的犯罪分子
		\item 拼接破碎的纸片
		\item 基于交通数据找到最优建路方案避免拥堵
		\item 通过影长比例计算当时的日地距离
		\item 预测世界杯冠军得主
		\item 帮助找到失事飞机的残骸
		\item $ \cdots $
	\end{itemize}
	
\end{frame}

\begin{frame}{\textbf{做数模需要学习什么}}\label{content}
	\begin{itemize}
	\item 模型/算法思想 
	
	\begin{itemize}
		\item 线性规划、整数规划、动态规划、组合优化$\cdots$
		\item 差值、拟合、回归$\cdots$
		\item 聚类分析、模糊综合评价、主成分分析法 $\cdots$
		\item 机器学习、人工神经网络、支持向量机$\cdots$
		\item 微分方程、差分方程$\cdots$
		\item 图与网络:图论、最短路径算法、最小生成树$\cdots$
		\item 遗传算法、模拟退火、粒子群$\cdots$
		\item $\cdots$
	\end{itemize}
	
	\item 编程/作图
	\begin{itemize}
		\item C/C++ 基础算法实现
		\item Matlab/Wolfram Mathematica 数值计算/作图基础
		\item R/Python$\cdots$
	\end{itemize}
	
	\item 写作/排版
	\begin{itemize}
		\item 科技写作规范和技巧
		\item  \LaTeX/\TeX 排版
	\end{itemize}
	\item $\cdots$
	\end{itemize}
	
\end{frame}

\begin{frame}{\textbf{报名方式}}\label{signin}
	\begin{exampleblock}{\textbf{这里没人?}}
		如果学长学姐不在C103,他们可能是在宣讲会现场,或者忙其他事情
	\end{exampleblock}
	
	\begin{alertblock}{\textbf{那怎么报名?}}
		\begin{itemize}
			\item 自取桌上的\textbf{创新中心宣传册}
			\item 撕下最后一页\textbf{报名表}, \textcolor{red}{*然后宣传册的其他部分可以带回去}
			\item 仔细填写
			\item 写完放在桌上就好了
			\item 坐等短信\textbf{面试通知}
		\end{itemize}
		\textcolor{red}{*如果来不及填写也可以带回去填完再抽空把报名表页交到C103} \\
		也可以在\textcolor{blue}{\hyperlink{qun}{数模组2015}}群里和我们谈笑风生!
	\end{alertblock}
	
	\begin{block}{\textbf{微信报名}}
		微信关注cippus\_ssdut,回复“姓名+学号+数模组+生日”报名 %\FA{}
	\end{block}
	
\end{frame}

\begin{frame}{\textbf{C103提供免费高速Wi-Fi}}
\begin{columns}
\column{0.6\textwidth}
{\LARGE 
    \begin{description}
        \item[WLAN] C103
        \item[密码] cippusc103
    \end{description}
}
\end{columns}
\end{frame}

\begin{frame}{\textbf{欢迎新生加入2015数模新生群}}
\framesubtitle{群号:178722688}
	\begin{figure}
		\includegraphics[width=.45\textwidth]{qq.jpg}
		\label{qun}
	\end{figure}

\end{frame}

\begin{frame}{\textbf{常见问题解答}}
\begin{alertblock}{\textbf{怎样加入数模组?}}
报名方式\textcolor{red}{\hyperlink{signin}{点这里}},成功报名后通过后续的面试/笔试的同学可以正式加入数模组大家庭。
\end{alertblock}

\begin{block}{\textbf{面试/笔试难吗?}}
我们不会刻意为难任何一位有志加入数模组的同学!但是我们当然希望大家能够不断提升自我,并且希望确保成员们保持对数模的热情,所以面试会问一些较为基础的编程知识和数学知识,以及一系列职业规划问题。如果对这些基础知识有所准备(也就是我们通常所说的\textcolor{blue}{预习}),对自己的人生规划有所思考,回答这些问题一定是轻而易举的。

\end{block}

\end{frame}

\begin{frame}
\begin{alertblock}{\textbf{需要预习哪些内容?}}
C语言和工科数学分析,面试可能涉及以下问题:
\begin{itemize}
    \item C语言:了解变量类型,3种基本程序结构的理解,基本的程序逻辑
    \item 工数:极限的定义,重要极限的求法等
\end{itemize}
预习这种事在保证质量的情况下,当然是多多益善的,有能力的同学大可突破这个范围要求,多学习对自身的帮助很大,我们也很喜欢学习态度认真的同学,即便不加入数模组。
\end{alertblock}

\begin{block}{\textbf{还没有加入数模组,能来C103参观/借书/找学长谈人生吗?}}
\textbf{没有问题,随时欢迎。}\\
学长学姐十分乐意为大家服务,能够向同学们介绍数模组我们也很高兴,有问题欢迎咨询。借书请报出具体书目,如果是教材可以直接找学长学姐们借,如果是书架上的其他文献,请向在场的学长学姐咨询相关事宜并留下纸质的借阅记录。如果学长学姐不在场,除了按\textcolor{blue}{\hyperlink{signin}{报名方法}}的指导填写报名表以外,请不要随意动他们桌上的任何东西。

\end{block}
\end{frame}

\begin{frame}
\begin{block}{\textbf{我高中的时候数学就很差,也能加入数模组?}}
做这个幻灯片的学姐在高中时代的数学一直在班级里吊车尾,常年不及格,一见数学题就害怕。高中数学和数学有很大的差别,高中时期数学差也可能和自己的心理状态、学习方法、教师水平甚至教育体制有很大关系,不妨在大学时期更加深入了解一下自己对数学的真正感情是什么。着手学习数模是一个很好的机会,数模也可以很大的改变自己的数学观,给自己一个机会,或许做多了你会发现对数学就并没有那么讨厌和害怕。
\end{block}
\begin{alertblock}{\textbf{我是调剂来的,对编程没兴趣想转专业,也能加入数模组?}}
当然可以。举个例子,数模组有一位09级的学姐,大一结束后转到了本部的数学专业,最后申请到了CMU的Master,但她一直是我们的一员,甚至曾从本部专程过来给新生讲例会。重要的是我们彼此对数模组的认同而不是身在何处。另外,一开始想转专业的同学,很多在一年的学习中对编程产生了兴趣,所以没有必要一开始就下非转专业不可的结论。
\end{alertblock}
\end{frame}

\begin{frame}
    \begin{exampleblock}{\textbf{与Oureda以及学生会、自强等组织社如何取舍?}}
    数模组并不排斥组员加入创新中心以外的任何其他组织,而且鼓励学生全面多元发展。我们支持组员在力所能及的情况下多参加其他组织及活动 ,但是毕竟每个人的精力有限,所以不鼓励逞强。比如挂名很多组织但在每个组织中的发展有限,最后疲惫不堪,这种情况我们也不希望看到。
    
    Oureda是朱明老师带领的一个十分优秀的实验室,目前发展蒸蒸日上,而且方向多元,经验丰富,设备充足,非常适合技术型人才的培养,更多细节见Oureda的宣传海报或者直接咨询朱明老师。而创新中心同样在发展中积累了得天独厚的经验和优势,从组织上来说,中心几乎是个纯学生组织,新生与前辈的交流更加顺畅且自由,更容易得到贴合入学新生视角的信息。当然以上叙述也是我主观的一家之言,仅供参考,对于迷茫的新生来说,多咨询这些组织相关和无关人员,以及亲自查看,自行判断,谨慎考虑。
    
    总之,无论如何选择,做到这些组织和课内学习、个人生活间的平衡十分重要,虽然优秀的人才没有进入数模组会是我们的遗憾,但是如果他们没有考虑自身情况量力而行,焦头烂额而无所获,更让我们心痛。
    \end{exampleblock}
\end{frame}

\begin{frame}
\begin{alertblock}{\textbf{我也好喜欢ACM组,应该选哪个?}}
因为照顾到每个人的精力有限,原则上不允许同时报这两个组。

这个问题可以首先试着了解一个ACM和MCM这两种有很大和区别的比赛,ACM对于算法的掌握、运用和实现有很高的要求,对于编程的帮助很大;而数模(MCM) 更侧重对模型和算法的创新和应用,有助于计算机科学的研究。当然它们的益处很大程度也是互通的,除此之外,ACM对于编程的训练要求更高一些,压力也相对更大。

在中心发展的早期和中期,这两个组从规模上来讲都是十分重要的大组,也是最频繁刷奖的两个组。这两个组的关系并不敌对,而是友好地合作互利共赢,ACM组的成员经常参加各类数模比赛,数模组成员也很热衷于参加ACM比赛,因为各组学习的内容对于这两种比赛都很实用。
\end{alertblock}

\begin{block}{\textbf{参加数模组会影响课内成绩么?}}
当然会,至于是正面的影响还是负面的影响取决于你的态度。优秀的学长学姐的辅导交流也是宝贵的资源,这里的学长学姐几乎都是拿过学习类奖学金的,每届都至少有一两个成绩是全专业前三的,学习氛围自然不错。数模固然重要而且有趣,但为此放弃学业还是不鼓励的。
\end{block}
\end{frame}

\begin{frame}
\begin{block}{\textbf{进入数模组是一种怎样的体验?(数模组的日常?)}}
\begin{itemize}
   \item 学长学姐每周组织一次例会,教同学们一些\hyperlink{content}{数模知识和技巧}(有时也有英语学习之类的彩蛋)
   \item 每个新生可以得到一位指定的学长学姐负责,带领新生完成大一一年的过渡
   \item 每周写周报与前辈们交流学习生活上遇到的困扰和收获,有时会布置每周作业帮助理解消化例会所讲的内容
   \item 几乎全员参加每年的省赛、国赛和美赛并得奖
   \item 每学期末有对成员的考核答辩,要求成员展示该学期的专业成绩、比赛得奖情况和数模及其他技术的作品。
   \item 可以使用C103的公共电脑、书籍和网络资源(对大一不能带电脑的新生来说很重要),C103也提供了一个安静的自习环境,同时可以拿到学长学姐的学习笔记、选课建议、实验室推荐等重要情报。
   \item 定期发起组内聚会(海烧或者日租之类的娱乐项目)
\end{itemize}
\end{block}
\end{frame}

\begin{frame}{\textbf{我们新生应该怎样预习?}}
  \framesubtitle{C语言}
  \begin{block}{}
  多跟着书上敲代码很重要,同时多看书理解每行代码的意义,关注结构,养成良好的coding习惯,不要浮躁。
  \end{block}
  \begin{description}
    \item[教材] \href{http://opac.lib.dlut.edu.cn/opac/item.php?marc_no=2010028166}{《C程序设计快速进阶大学教程》}
    \item[参考书籍\footnote{为避免不必要的困扰,请\textbf{不要}从谭浩强的书入手}] \href{http://opac.lib.dlut.edu.cn/opac/item.php?marc_no=2007039426}{《C语言教程:programming in C》}、\href{http://opac.lib.dlut.edu.cn/opac/item.php?marc_no=2003046536}{《C程序设计语言》}、\href{http://opac.lib.dlut.edu.cn/opac/item.php?marc_no=2008007312}{《C和指针》} 
    \item[推荐顺序] 遵从教材第5-13章的顺序 
    \item[编译器] 
    \begin{itemize}
     \item Visual C++/Visual Studio: VC是上机用的编译器,语法和VS相同,但win7以上可能安装会遇到问题,所有为了模拟上机可以安装VS,VS是十分强大的IDE
     \item CodeBlocks\textcolor{red}{+MinGW}:ACM钦定的优秀的轻量编译器
     \item C4droid: Android平台上的简易C/C++编译器,方便没带电脑的新生
    \end{itemize}
  \end{description}
\end{frame}

\begin{frame}{\textbf{我们新生应该怎样预习?}}
\framesubtitle{C语言}
\begin{description}
    \item[debug建议]
    \begin{itemize}
    \item 尽量使用英文路径,编译完成后双击错误信息可直接跳转到有问题的代码
    \item 复制错误信息,用百度或者谷歌搜索
    \end{itemize}
    \item[在线资源] \begin{itemize}
    \item \url{http://codepad.org/}、\url{http://tool.runoob.com/index.php/Home/Index/compile/language/c}:在线编译器,可在任意终端的浏览器使用,在手机上也能编程,分享代码便捷
    \item \url{http://www.runoob.com/cprogramming/c-tutorial.html}:C语言在线教程
    \item \href{https://www.coursera.org/learn/jisuanji-biancheng}{《计算导论与C语言基础》}和\href{https://www.coursera.org/learn/c-chengxu-sheji}{《C程序设计进阶》}: Coursera平台上的在线课程,由北京大学提供,有配套的作业、练习和讨论平台\footnote{Android和iOS都有Coursera的客户端}
    \item \url{https://en.wikibooks.org/wiki/C_Programming}
    \end{itemize}
\end{description}
\end{frame}

\begin{frame}{\textbf{我们新生应该怎样预习?}}
\framesubtitle{工科数学分析}
\begin{block}{}
工数和本部其他专业学的微积分、高等数学两门课程同气连枝,在某些章节略有不同,上学期只涉及一元函数的极限、微分、积分等内容,大概在期中阶段学习可能会遇到瓶颈,吃透习题有助于加深理解突破障碍,习题以近年教辅为主,吉米多维奇的数分如果实在想做挑些精选的做,不用全做。
\end{block}
\begin{description}
    \item[教材] \href{http://opac.lib.dlut.edu.cn/opac/item.php?marc_no=2008000575}{《工科数学分析》}
    \item[官方教辅] \href{http://opac.lib.dlut.edu.cn/opac/item.php?marc_no=2010052839}{《工科数学分析同步辅导》}、历年期末考题和模拟题\textcolor{red}{(大红本)}
    \item[参考书籍] \href{http://opac.lib.dlut.edu.cn/opac/item.php?marc_no=2007023436}{《高等数学(上册)》}、\href{http://opac.lib.dlut.edu.cn/opac/item.php?marc_no=2009018280}{《微积分(上册)》}、\href{http://www.cengagebrain.com/content/stewart97815_0538497815_01.01_toc.pdf}{\textit{Calculus}}\footnote{这本书在市面上也有\href{http://item.jd.com/10075433.html}{中文版},但本书作为经典教材,文字和内容都浅显易懂,适合基础薄弱者参考,也适合作为原版教材的入门读物。}
    \item[推荐顺序] 根据教材顺序预习1-4.3
    
\end{description}
\end{frame}

\begin{frame}{\textbf{我们新生应该怎样预习?}}
\framesubtitle{工科数学分析}
\begin{description}
 \item[工具] \href{http://www.wolframalpha.com/}{Wolfram Alpha}\footnote{Android、iOS、WP上均有客户端,可以随时随地用来解题}: 强大的数学搜索引擎,可以通过搜索直接得到题目的答案和题解,也能用来制作简易的绘图,搜索概念、定义和数据。
 \item[在线资源] \begin{itemize}
 \item \url{https://en.wikibooks.org/wiki/Calculus}
 \item \href{http://open.163.com/special/sp/singlevariablecalculus.html}{麻省理工学院公开课:单变量微积分}
 \item \href{https://www.coursera.org/learn/calculus1}{微积分基础:俄亥俄州立大学} 

 \end{itemize}
\end{description}
\end{frame}




\end{document}