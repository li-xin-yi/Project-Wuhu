\documentclass{exam}
\usepackage{CJKutf8}
\usepackage{amsmath}
\usepackage{amssymb} 
\newif\ifprint
\unframedsolutions
\renewcommand{\solutiontitle}{\noindent\textbf{解: }\noindent}
%\printtrue %选择题答案/printfalse时打印,/printtrue时隐藏
\printfalse
\usepackage{ulem}
\newcommand{\blank}[2][1cm]{\uline{\makebox[#1][c]{%
  \ifprint
    \phantom{#2}%
  \else
    #2%
  \fi}}}
\printanswers %答题答案去掉后隐藏

\begin{document}

\begin{CJK*}{UTF8}{gbsn}
\section*{选择题}
    \begin{questions}
    \question 以下关于联合熵的命题\blank{}恒为真
    \begin{choices}
        \choice $H({X_1}, \ldots ,{X_n}) = H({X_1}) +  \cdots  + H({X_n})$
        \choice $H({X_1}, \ldots ,{X_n}) \le H({X_1}) +  \cdots  + H({X_n})$
        \choice $H({X_1}, \ldots ,{X_n}) \ge H({X_1}) +  \cdots  + H({X_n})$
        \choice $H({X_1}, \ldots ,{X_n}) \ne H({X_1}) +  \cdots  + H({X_n})$\\
    \end{choices}
    
    \question $F$是一个多对一的函数,则以下为真的是\blank{}
    \begin{choices}
        \choice $H(F(X)) = H(X)$
        \choice $H(F(X)) > H(X)$
        \choice $H(F(X)) < H(X)$
        \choice 都有可能\\
    \end{choices}
    
    \question $X$和$Y$同分布且概率独立,则以下命题\blank{}恒为真
    \begin{choices}
        \choice $H(X,Y,Z) - H(X,Y) = H(X,Z) - H(X)$
        \choice $H(X,Y,Z) - H(X,Y) \le H(X,Z) - H(X)$
        \choice $H(X,Y,Z) - H(X,Y) \ge H(X,Z) - H(X)$
        \choice $H(X,Y,Z) - H(X,Y) \ne H(X,Z) - H(X)$\\
    \end{choices}
    
    \question 对不同的$i$,$\left( {{X_i},{Y_i}} \right)$之间是概率独立的离散型联合随机变量,概率分布为$P(X,Y)$,$P(X)$和$P(Y)$分别为各个$X_i$和$Y_i$的概率分布,$i = 1,2, \ldots ,n$.根据大数定律,当$n$趋于无穷大时,随机变量$\frac{1}{n}\log \frac{{P\left( {{X_1}, \ldots ,{X_n}} \right)P\left( {{Y_1}, \ldots ,{Y_n}} \right)}}{{P({X_1},{Y_1}, \ldots {X_n},{Y_n})}}$的极限是\blank{}
    \begin{choices}
        \choice $H(X|Y) - H(Y|X)$
        \choice $H(X|Y) + H(Y|X)$
        \choice $I(X;Y)$
        \choice $-I(X;Y)$\\
    \end{choices}
    
    \question一个二元对称无记忆离散信道的容量为0.8比特,信道编码采用二进制形式,可以表达为16种序列,要使译码的差错概率能够任意地接近于0,信道码字的长度最短不能低于\blank{}
    \begin{choices}
        \choice 4位
        \choice 5位
        \choice 6位
        \choice 7位
    \end{choices}
\end{questions}

\newpage

\section*{证明题}
    以下出现的随机变量$X$、$Y$和$Z$都是离散随机变量
    \begin{questions}
    \question 请证明关于离散随机变量$X$的信息熵$H(X)$是凹函数
    \question 陈述信息处理不等式
    \question 证明上述的信息处理不等式
    \question 规定条件互信息量为$I(X;Y|Z)$, $I(X;Y|Z) = H(X|Z) - H(X|Y,Z)$, 请证明$I(X;Y|Z)$对于$X$和$Y$满足对称性,即$I(X;Y|Z)=I(Y;X|Z)$
    \question 以上符号含义不变请证明$I(X;Z|Y)-I(Y;Z|X)=I(X;Z)-I(Y;Z)$
    \end{questions}

\newpage

\section*{计算题}
    \begin{questions}
    \question 高斯信道$Y=hX+Z$,$h$为信号放大倍数(信号增益),输入信号$X$的功率为$P$,$Z$为均值为零方差为$a^2$的随机噪声,推导该高斯信道的容量
    \begin{center}
    \begin{picture}(220,80)(-20,0)
        \put(-20,10){\circle{20}}
        \put(100,10){\circle{20}}
        \put(180,10){\circle{20}}
        \put(100,70){\circle{20}}
        \put(20,3){\framebox(40,14){$h$}}
        \put(60,10){\vector(1,0){30}}
        \put(-10,10){\vector(1,0){30}}
        \put(110,10){\vector(1,0){60}}
        \put(100,60){\vector(0,-1){40}}
        \put(-25,7){$X$}
        \put(96,7){$+$}
        \put(176,7){$Y$}
        \put(96,67){$Z$}
    \end{picture}
    \end{center}
    \question 将两个高斯信道如图串联,第一级增益为$h_1$,第二级增益为$h_2$,两个信道的噪声$Z_1$和$Z_2$的方差分别为$a^2$和$b^2$,输入信号$X$的功率仍然是$P$,求信道容量
    \begin{center}
    	\begin{picture}(330,80)(-20,0)
    	\put(-20,10){\circle{20}}
    	\put(100,10){\circle{20}}
    	\put(220,10){\circle{20}}
    	\put(300,10){\circle{20}}
    	\put(100,70){\circle{20}}
    	\put(220,70){\circle{20}}
    	\put(20,3){\framebox(40,14){$h_1$}}
    	\put(140,3){\framebox(40,14){$h_2$}}
    	\put(60,10){\vector(1,0){30}}
    	\put(-10,10){\vector(1,0){30}}
    	\put(110,10){\vector(1,0){30}}
    	\put(180,10){\vector(1,0){30}}
    	\put(230,10){\vector(1,0){60}}
    	\put(100,60){\vector(0,-1){40}}
    	\put(220,60){\vector(0,-1){40}}
    	\put(-25,7){$X$}
    	\put(96,7){$+$}
    	\put(216,7){$+$}
    	\put(94,67){$Z_1$}
    	\put(214,67){$Z_2$}
    	\put(296,7){$Y$}
    	\end{picture}
    \end{center}
    \question 分别考虑两种情况下上题信道容量$C$的极限 (1)$h_2$固定,$h_1$趋于无穷大\\(2)$h_1$固定,$h_2$趋于无穷大
    \question 以上符号含义不变,求如图并联的高斯信道的信道容量
    \begin{center}
    \begin{picture}(320,200)
	    \put(10,100){\circle{20}}
	    \put(20,100){\line(1,0){60}}
	    \put(80,70){\line(0,1){60}}
	    \put(80,70){\vector(1,0){30}}
	    \put(80,130){\vector(1,0){30}}
	    \put(110,123){\framebox(40,14){$h_1$}}
	    \put(110,63){\framebox(40,14){$h_2$}}
	    \put(150,70){\vector(1,0){30}}
	    \put(150,130){\vector(1,0){30}}
	    \put(190,70){\circle{20}}
	    \put(190,10){\circle{20}}
	    \put(190,20){\vector(0,1){40}}
	    \put(190,130){\circle{20}}
	    \put(190,190){\circle{20}}
	    \put(190,180){\vector(0,-1){40}}
	    \put(200,70){\line(1,0){30}}
	    \put(200,130){\line(1,0){30}}
	    \put(230,70){\line(0,1){60}}
	    \put(230,100){\vector(1,0){60}}
	    \put(300,100){\circle{20}}
	    \put(6,97){$X$}
	    \put(186,127){$+$}
	    \put(186,67){$+$}
	    \put(184,187){$Z_1$}
	    \put(184,7){$Z_2$}
	    \put(296,97){$Y$}
    \end{picture}
    \end{center}
    \end{questions}

\newpage

\section*{计算题}
    \begin{questions}
    \question 设二元对称离散无记忆信息的传输差错概率为$p$,记为$BSC(p)$,请计算其容量$C$
    \question 将$N$个$BSC(p)$信道串联,结果得到一个等效的BSC信道.计算其信道容量$C$(用$N$和$p$表示)
    \question 将$N$个$BSC(p)$信道串联且这N个信道相互独立(无串扰),结果得到一个输入和输出为N维的二进制向量的矢量信道,并请计算其容量$C$(用$N$和$p$表示)
    \end{questions}
\end{CJK*}
\end{document}