
\documentclass{exam}
\usepackage{CJKutf8}
\usepackage{CJKnumb}
\usepackage{paralist}
\usepackage{subfigure}
\usepackage[usenames,dvipsnames]{xcolor}
\usepackage[colorlinks,linkcolor=red,urlcolor=Orchid]{hyperref}
\renewcommand{\thepartno}{\arabic{partno}}
\renewcommand\thesection{\CJKnumber{\arabic{section}}}
\pointname{分}
\newif\ifprint
\unframedsolutions
%\renewcommand{\solutiontitle}{\noindent\textbf{解: }\noindent}
%\printtrue %选择题答案/printfalse时打印,/printtrue时隐藏
\printfalse
\usepackage{ulem}
\newcommand{\blank}[2][1cm]{\uline{\makebox[#1][c]{%
			\ifprint
			\phantom{#2}%
			\else
			#2%
			\fi}}}
\pagestyle{headandfoot}
\cfoot{\thepage$/$\numpages} 
\header{2015/6/27}{操作系统2015}{\href{https://github.com/Lixinyi-DUT/Project-Wuhu}{芜湖计划}}
\headrule


\begin{CJK*}{UTF8}{gbsn}
\begin{document}
\section{选择题(20分)}
	\begin{questions}
		\question[2]并发进程中访问相同变量的程序段,被称为\blank{}
		\begin{choices}
			\choice 临界区
			\choice 临界资源
			\choice 缓冲区
			\choice 原语
		\end{choices}
		
		\question[2]进程在执行期间发生了缺页中断,经操作系统处理后,应让其接着执行\blank{}
		\begin{choices}
			\choice 被中断的前一条
			\choice 被中断的那一条
			\choice 被中断的后一条
			\choice 启动时的第一条
		\end{choices}
		
		\question[2]在进程状态转换时,\blank{}转换是不可能的
		\begin{choices}
			\choice 就绪$\to$运行
			\choice 运行$\to$就绪
			\choice 运行$\to$阻塞
			\choice 阻塞$\to$运行
		\end{choices}
		
		\question[2]进程和程序的本质区别之一是\blank{}
		\begin{choices}
			\choice 前者分时使用CPU,后者独占CPU
			\choice 前者存储在内存,后者存储在外存
			\choice 前者在同一个文件中,后者在多个文件中
			\choice 前者是动态的,后者是静态的
		\end{choices}
		
		\question[2]动态定位是在\blank{}中进行的
		\begin{choices}
			\choice 编译过程
			\choice 装入过程
			\choice 修改过程
			\choice 执行过程
		\end{choices}
		
		\question[2]下面的\blank{}页面淘汰算法有时会产生Belady异象
		\begin{choices}
			\choice 先进先出(FIFO)
			\choice 最近最少使用(LRU)
			\choice 最不经常使用(LFU)
			\choice 理想型(OPT)
		\end{choices}
		
		\question[2]死锁的预防是通过破坏死锁产生的四个必要条件来实现的,下列方法中\blank{}破坏了"环路条件"
		\begin{choices}
			\choice 银行家算法
			\choice 资源有序分配
			\choice 一次性分配策略
			\choice SPOOLING技术
		\end{choices}
		
		\question[2]批量处理系统的主要目的是尽量提高系统的吞吐量,为此,应优选择\blank{}运行
		\begin{choices}
			\choice 使用户比较满意的作业
			\choice 运算量大的作业
			\choice 耗时较短的作业
			\choice 优先级较高的作业
		\end{choices}
		
		\question[2]有一磁盘,共10个柱面,每个柱面20个磁道,每个盘面分成16个扇区,采用位示图对其存储空间进行管理,如果字长是16个二进制位,那么示图共需\blank{}个字
		\begin{choices}
			\choice 200
			\choice 128
			\choice 256
			\choice 100
		\end{choices}
		
		\question[2]实现虚拟存储器的目的是\blank{}
		\begin{choices}
			\choice 扩充物理主存
			\choice 逻辑上扩充主存
			\choice 逻辑上扩充外存
			\choice 以上都不对
		\end{choices}
	\end{questions}
	
\section{简答题(25分)}
	\begin{questions}
		\question[5] 请简述文件系统的分层组织结构
		\vspace{1cm}
		\question[5]请列举文件系统中目录的几种典型结构,并从文件命名、文件分组、效率等角度,阐述每种目录结构的优劣
		\vspace{1.5cm}
		\question[5]磁盘调度策略有哪几种,请简要说明各磁盘调度算法的原理
		\vspace{1.2cm}
		\question[5]请简要描述在虚存机制下页面异常处理的基本流程
		\vspace{1.3cm}
		\question[5]试从调度性、并发性、拥有资源及系统开销几个方面,对进程和线程进行比较
		\vspace{1cm}
	\end{questions}

\section{计算题(55分)}
    \begin{questions}
    	
	\question[10] 有五个进程A、B、C、D、E几乎同时到达(任务到达的先后顺序为C,D,B,E,A),估计的运行时间分别为2、4、6、8、10分钟,它们的优先数分别为1、2、3、4、5(1为最优先级)。对于下面每种调度算法,分别计算任务的平均周转时间:
	\begin{inparaenum}[ (1)]
		\item 最高优先级优先
		\item 时间片轮转
		\item FIFO
		\item 短作业优先
	\end{inparaenum}
	\vspace{2.5cm}
	
	\question[10] 在一个页式系统中,页面的大小为1KB,地址寄存器的字长为20位,现有一长度为4KB的用户程序,有4个页面分别被分配在内存的页框号为10、14、15和18的物理页中。当程序中的访问地址分别为2058、4011、5890时,说明各自的地址转换结果。
	
	\vspace{2cm}
	
	\question[10]系统中有五个进程,分别为P1,P2,P3,P4,P5,四类资源分别为R1、R2、R3、R4。某一时刻系统资源向量$A=(1,2,3,0)$
	\begin{parts}
		\part 试用银行家算法判断系统当前状态是否安全
		\part 当进程P3提出对R3的剩余请求时,是否能满足它?请详细说明理由。
		\part 系统初始配置的各类资源分别是多少?
	\end{parts}
	\begin{table}[htb]
			\begin{center}
		    \subtable[最大需求表$Q$]{
		    	\begin{tabular}{|c|c|c|c|c|}
		    	\hline
		    	& R1 & R2 & R3 & R4 \\
		    	\hline
		    	P1 & 1 & 2 & 1 & 2 \\
		    	\hline
		    	P2 & 1 & 7 & 5 & 0 \\
		    	\hline
		    	P3 & 2 & 3 & 5 & 6 \\
		    	\hline
		    	P4 & 0 & 8 & 5 & 2 \\
		    	\hline
		    	P5 & 0 & 6 & 3 & 6 \\
		    	\hline
		    \end{tabular}
	    	}
	    	\qquad \qquad
	    	\subtable[已分配表格$U$]{
		    \begin{tabular}{|c|c|c|c|c|}
		    	\hline
		    	& R1 & R2 & R3 & R4 \\
		    	\hline
		    	P1 & 0 & 0 & 1 & 2 \\
		    	\hline
		    	P2 & 1 & 0 & 0 & 0 \\
		    	\hline
		    	P3 & 1 & 1 & 4 & 4 \\
		    	\hline
		    	P4 & 0 & 6 & 2 & 2 \\
		    	\hline
		    	P5 & 0 & 0 & 1 & 4 \\
		    	\hline
		    \end{tabular}
	    	}
		    \end{center}
	\end{table}
	\newpage
	\question[15]某工厂有两个生产车间和一个装配车间,两个生产车间分别生产A、B两种零件、装配车间的任务是把A、B两种零件组装成产品。两个车间每生产一个A零件和一个B零件,然后都要分别把它们送到装配车间的货架F1,F2上,F1存放零件A,F2存放零件B,F1、F2的容量均可以存放10个零件。装配工人每次从货架上取一个A零件和一个B零件,然后组装成产品。请用P、V操作对生产装配过程进行正确管理。
	\vspace{6cm}
	
	\question[10]在页式虚拟存储系统中,某个进程被分配有4个物理页,进程刚开始时,物理页内容均为空。若该进程按如下序列访问程序中的页:\\
	5,3,6,4,3,5,1,4,2,5,6,4,2,5,1\\
	试计算采用如下页置换算法时的缺页次数,并给出各种情况下的具体页面置换情况图示。
	\begin{inparaenum}[ (1)]
		\item 采用FIFO算法
		\item 采用LRU算法
		\item 采用OPT算法
	\end{inparaenum}
	\clearpage
\end{questions}
\end{CJK*}
\end{document}