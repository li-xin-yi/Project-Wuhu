\documentclass[a4paper]{exam}
\usepackage{CJKutf8}
\usepackage{amsmath}
\usepackage{CJKnumb}
\usepackage{paralist}
\usepackage[usenames,dvipsnames]{xcolor}
\usepackage[colorlinks,linkcolor=red,urlcolor=LimeGreen]{hyperref}
\renewcommand{\thequestion}{\bf{\CJKnumber{\arabic{question}}}}
\renewcommand{\questionlabel}{\thequestion}
\newif\ifprint
%\printtrue %选择题答案/printfalse时打印,/printtrue时隐藏
\printfalse
\printanswers
\usepackage{ulem}
\newcommand{\blank}[2][1cm]{\uline{\makebox[#1][c]{%
			\ifprint
			\phantom{#2}%
			\else
			#2%
			\fi}}}
\pagestyle{headandfoot}
\cfoot{\thepage$/$\numpages} 
\header{2015/6/28}{线性代数2015}{\href{https://github.com/Lixinyi-DUT/Project-Wuhu}{芜湖计划}}
\headrule

\begin{document}

\begin{CJK*}{UTF8}{gbsn}

\begin{questions}
\question {\bf{{\large 填空题}}}
    \renewcommand{\partlabel}{\thepartno.}
    \renewcommand{\thepartno}{\arabic{partno}}
	\begin{parts}
		\part 设 ${\bf{a}}{\rm{ = }}{\left[ {1,3,4} \right]^T}$, ${\bf{b}} = {\left[ {1, - 1,2} \right]^T}$,则${({\bf{a}}{{\bf{b}}^T})^{80}}=$\blank{}
		
	    \part 设$A$为三阶方阵,$\left| {A} \right| = 3$,则$\left| {{{{A}}^*} + {{A}^{ - 1}}} \right| = $\blank{}
	    
	    \part 设${{\bf{a}}_1}$,${{\bf{a}}_2}$,${{\bf{a}}_3}$都是三元列向量,$A = [{{\bf{a}}_1},{{\bf{a}}_2},{{\bf{a}}_3}]$,$B = [{{\bf{a}}_3},{{\bf{a}}_2}, - {{\bf{a}}_1}]$,则$\left| {A + B} \right| = $\blank{}
	    
	    \part 设方阵$A$满足$A^2-A=0$,则${(A-3E)}^{-1}=$\blank{}
	    
	    \part 设$A$为三阶方阵,$r(A) = 2$,${{\bf{u}}_1} = {[1, - 1,2]^T}$和${{\bf{u}}_2} = {[0,0,1]^T}$是方程组$A{\bf{x}} = {\bf{b}}$的两个解,则方程组$A{\bf{x}} = {\bf{b}}$的通解为\blank{}
	    
	    \part 设向量${{\bf{a}}_1}$,${{\bf{a}}_2}$,${{\bf{a}}_3}$,${{\bf{a}}_4}$线性无关,${{\bf{b}}_1} = {{\bf{a}}_1} + {{\bf{a}}_2}$,${{\bf{b}}_2} = 2{{\bf{a}}_2} + {{\bf{a}}_3}$,${{\bf{b}}_3} = 2{{\bf{a}}_3} + {{\bf{a}}_4}$,${{\bf{b}}_4} = 2{{\bf{a}}_4} + k{{\bf{a}}_1}$,则向量组${{\bf{b}}_1}$,${{\bf{b}}_2}$,${{\bf{b}}_3}$,${{\bf{b}}_4}$线性无关的充要条件是$k$满足\blank{}
	    
	    \part 向量${\bf{b}} = \left[ {\begin{array}{*{20}{c}} 3\\	1\\	7 \end{array}} \right]$在基${{\bf{a}}_1} = \left[ {\begin{array}{*{20}{c}} 1\\ 0\\ 2 \end{array}} \right]$,${{\bf{a}}_2} = \left[ {\begin{array}{*{20}{c}} 2\\ 1\\ 4 \end{array}} \right]$,
	    ${{\bf{a}}_3} = \left[ {\begin{array}{*{20}{c}} 3\\ 1\\ 5 \end{array}} \right]$ 下的坐标向量为\blank{}
	    
	    \part 设$A$为三阶方阵,$\left| A \right| = 0$,$tr(A)=1$,$r(3E+A)=2$,则$\left| A+E \right| =$\blank{}
	    
	    \part 二次型$f({x_1},{x_2},{x_3}) = x_1^2 + kx_2^2 + 3x_3^2 + 2{x_1}{x_2} + 2{x_1}{x_3} + 2k{x_2}{x_3}$为正定二次型的充要条件是$k$满足\blank{}
	    
	    \part 设$A = \left[ {\begin{array}{*{20}{c}} k&3&3&3\\ 3&k&3&3\\ 3&3&k&3\\ 3&3&3&k \end{array}} \right]$
	    ,$r(A^*)=1$,则$k=$\blank{}
	    \end{parts}
	    
	    \vspace{1cm}
	    \renewcommand{\partlabel}{(\thepartno)}
	    \question 
	    \begin{parts}
	    	\part 求过点$P_0(1,1,0)$且平行于向量$\vec a = \vec i + 2\vec j + \vec k$和$\vec b = 2\vec i + \vec j$的平面方程
	    	\part 将直线$L$的一般式方程$\begin{cases}
	       x+y-z&=0  \\
	       -x+y-2z&=2 
	    	\end{cases}$
	    	化为对称式方程
	    \end{parts}
	    \vspace{2.5cm}
 
		\question 计算行列式$\left| {\begin{array}{*{20}{c}}
			1&2&2&2&2\\
			2&5&1&1&1\\
			2&1&5&1&1\\
			2&1&1&5&1\\
			2&1&1&1&5
			\end{array}} \right|$
		
		\vspace{2.5cm}
	
	\question 已知向量组${{\bf{a}}_1} = {[1, -2,0,1]^T}$,${{\bf{a}}_2} = {[-1, 3,2,-2]^T}$,${{\bf{a}}_3} = {[2, -3,2,1]^T}$,${{\bf{a}}_4} = {[0,1,3,1]^T}$,${{\bf{a}}_5} = {[3,-8,-5,k]^T}$的秩为3,求$k$及列向量组的一个极大无关组并将其他向量用该极大无关组线性表示。
	
	\vspace{3cm}
	
	\question 设$A = \left[ {\begin{array}{*{20}{c}}
		2&{ - 1}&0\\
		0&0&{ - 1}\\
		1&0&3
		\end{array}} \right]$,$X = {A^{ - 1}}X + E$,求$X$
	
	\vspace{2.5cm}
	
	\question 当$a$,$b$满足什么条件时,方程组$\begin{cases}
	x_1-x_2-2x_3 &=0\\
	2x_1-x_2-5x_3 &=b\\
	-2x_1+3x_2+ax_3 &=1
	\end{cases}$
	\begin{inparaenum}[ (1)]
		\item 有唯一解
		\item 无解
		\item 无穷多解
	\end{inparaenum}?
	并在有无穷多解时,求该方程组通解
	
	\vspace{3.5cm}
	
	\question 设$A = \left[ {\begin{array}{*{20}{c}}
		{ - 1}&{ - 1}&{ - 1}\\
		{ - 1}&a&1\\
		{ - 1}&1&{ - 1}
		\end{array}} \right]$与 $B = \left[ {\begin{array}{*{20}{c}}
		{ - 2}&{}&{}\\
		{}&{ - 2}&{}\\
		{}&{}&b
		\end{array}} \right]$相似
	\begin{parts}
		\part 求$a$和$b$
		\part 求正交矩阵$Q$,使$Q^{-1}AQ=B$
		\part 设${\bf{u}} = {[x,y,z]^T}$,试问方程${{\bf{u}}^T}A{\bf{u}}=1$表示什么曲面
	\end{parts}
	
	\vspace{4cm}
	
	\question 设$A$为三阶方阵,${\alpha _1}$,${\alpha _2}$分别为$A$的特征值1和-1对应的特征向量。 $A\alpha_3=\alpha_1+\alpha_2$,证明${\alpha _1}$,${\alpha _2}$,${\alpha _3}$线性无关
	
	\vspace{4.5cm}
	
	\question 设$A$为三阶方阵,$A$的每个元素都与其对应的代数余子式相等,$\left| A \right| \ne 0$.\\
	证明:$A$为正交矩阵.
	
\end{questions}
\newpage
\end{CJK*}


\end{document}