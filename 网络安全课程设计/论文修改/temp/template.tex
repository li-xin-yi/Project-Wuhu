%%%%%%%%%%%%%%%%%%%%%%% file template.tex %%%%%%%%%%%%%%%%%%%%%%%%%
%
% This is a general template file for the LaTeX package SVJour3
% for Springer journals.          Springer Heidelberg 2010/09/16
%
% Copy it to a new file with a new name and use it as the basis
% for your article. Delete % signs as needed.
%
% This template includes a few options for different layouts and
% content for various journals. Please consult a previous issue of
% your journal as needed.
%
%%%%%%%%%%%%%%%%%%%%%%%%%%%%%%%%%%%%%%%%%%%%%%%%%%%%%%%%%%%%%%%%%%%
%
% First comes an example EPS file -- just ignore it and
% proceed on the \documentclass line
% your LaTeX will extract the file if required
%
%
\RequirePackage{fix-cm}
%
\documentclass{svjour3}                     % onecolumn (standard format)
%\documentclass[smallcondensed]{svjour3}     % onecolumn (ditto)
%\documentclass[smallextended]{svjour3}       % onecolumn (second format)
%\documentclass[twocolumn]{svjour3}          % twocolumn
%
\smartqed  % flush right qed marks, e.g. at end of proof
%
\usepackage{graphicx}
%
% \usepackage{mathptmx}      % use Times fonts if available on your TeX system
%
% insert here the call for the packages your document requires
%\usepackage{latexsym}
% etc.
%
% please place your own definitions here and don't use \def but
% \newcommand{}{}
%
% Insert the name of "your journal" with
% \journalname{myjournal}
%
\begin{document}

\title{Cost-sensitive Algorithm
	with Local and Global Consistency%\thanks{Grants or other notes
%about the article that should go on the front page should be
%placed here. General acknowledgments should be placed at the end of the article.}
}
\subtitle{}

%\titlerunning{Short form of title}        % if too long for running head

\author{Weifeng Sun       \and
        Jianli Sun %etc.
}

%\authorrunning{Short form of author list} % if too long for running head

\institute{W. Sun \at
              School of Software, Dalian University of Technology \\
              DaLian, 116620 - China \\
              \email{}           %  \\
%             \emph{Present address:} of F. Author  %  if needed
           \and
           J. Sun \at
           School of Software, Dalian University of Technology\\
           DaLian, 116620 - China\\
           \email{sjl\_dlut@163.com}
}

\date{Received: date / Accepted: date}
% The correct dates will be entered by the editor


\maketitle

\begin{abstract}
Assuming that misclassification cost among different categories equals to each other, traditional graph based semi-supervised classification algorithms pursuit high classification accuracy. However, in many practical problems, misclassifying one category as another always leads to higher (or lower) cost than that in turn, such that, higher classification accuracy generally not means lower cost, which is more important in these problems. Cost-sensitive classification methods enable classifiers to pay more attention to data samples with higher cost, and then attempt to get lower cost by ensuring higher classification accuracy of the category with higher cost. In this paper, we bring cost sensitivity to local and global consistency (LGC) classifiers, and propose the CS-LGC (cost-sensitive LGC) methods, which can make better use of semi-supervised classification algorithms, and ensure high classification accuracy on the basis of reducing overall cost. At the same time, since the improved algorithm may bring some problems due to unbalanced data account, we introduce a SMOTE algorithm for further optimization. Experimental results of bank loan and diagnosis problems verify the effectiveness of CS-LGC.
\keywords{Graph based semi-supervised classification \and Cost-sensitive \and Rescale \and SMOTE}
% \PACS{PACS code1 \and PACS code2 \and more}
% \subclass{MSC code1 \and MSC code2 \and more}
\end{abstract}

\section{Introduction}
\label{intro}

Recently, machine learning classification algorithms have achieved sound development in multimedia identification, health care, finance, and many other applications. Among them, Graph-based Semi-Supervised Classification based on perfect graph theory, which is relatively straightforward and easy to understand, has become one of the hotest research focuses. This method can make better use of the information from label data and a great deal of data distribution information mined from unlabeled data, then solves the problem of the less labeled data and the huge labeling costs.

High classification accuracy is the target of GSSC algorithm, which assumes that the cost of misclassification among different categories is the same. However, in many practical problems,  the costs of category classifications is different, especially in the fields of medical service, finance and network security and so on.

For example, in the bank loan problem, banks decide whether to approve the loan based on the loan applicants' personal information, which can be viewed as a binary classification problem. In the process of approving a bank loan, the cost of approving an applicant who is unable to repay the loan is much greater than that of approving an applicant who can repay the loan. Similarly, in the disease diagnosis, the cost of misdiagnose unhealthy patients as healthy is much greater than that of misdiagnose healthy patients as unhealthy. 
\section{Section title}
\label{sec:1}
Text with citations  and %\cite{urner2011access} \cite{zhou2010semi} \cite{zhu2009introduction}
\subsection{Subsection title}
\label{sec:2}
as required. Don't forget to give each section
and subsection a unique label (see Sect.~\ref{sec:1}).
\paragraph{Paragraph headings} Use paragraph headings as needed.
\begin{equation}
a^2+b^2=c^2
\end{equation}

%

%\begin{acknowledgements}
%If you'd like to thank anyone, place your comments here
%and remove the percent signs.
%\end{acknowledgements}

% BibTeX users please use one of
%\bibliographystyle{spbasic}      % basic style, author-year citations
%\bibliographystyle{spmpsci}      % mathematics and physical sciences
%\bibliographystyle{spphys}       % APS-like style for physics
%\bibliography{}   % name your BibTeX data base
\bibliographystyle{ieeetr}
\bibliography{myref}

% Non-BibTeX users please use
\end{document}
% end of file template.tex

